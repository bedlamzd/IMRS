%! Author = bedlamzd
%! Date = 15.03.2021

\documentclass[14pt]{extarticle}

% Preamble
\include{preamble}

% Document
\begin{document}

    \include{title_page}

    \section*{Цель работы}
    \begin{itemize}
        \item Средствами симулятора собрать модель схвата манипулятора
        \item Использовать сенсор для обнаружения объектов в схвате
        \item Написать скрипт активации схвата на основе данных датчика
    \end{itemize}

    \section*{Ход работы}

    Соберём из простейших элементов примитивный схват манипулятора. Для этого воспользуемся кубами и линейными
    актуаторами. Соответственно кубы представляют собой конкретные детали схвата, а актуаторы это некоторый виртуальный
    линейный мотор. При этом актуатору не обязательно находиться в непосредственном контакте с деталью, поскольку у него
    нет физических свойств.

    Добавив на левую часть схвата сенсор получим модель изображённую на рисунке~\ref{pic:wip gripper}.

    \begin{figure}[H]
        \centering
        \includegraphics[width=0.75\textwidth]{gripper1.png}
        \caption{Модель схвата}
        \label{pic:wip gripper}
    \end{figure}

    \begin{figure}[H]
        \centering
        \includegraphics{tree.png}
        \caption{Дерево модели в процессе работы}
        \label{pic:wip tree}
    \end{figure}

    \begin{figure}[H]
        \centering
        \includegraphics[width=0.4\textwidth]{joint_settings.png}
        \caption{Параметры линейного актуатора}
        \label{pic:joint settings}
    \end{figure}

    Добавим в сцену объект для обнаружения датчиком и напишем простой скрипт (Приложение~\ref{code:script}), который
    считывает с датчика данные и, как только в его пределах видимости появляется деталь, подаёт команду на схват.

    \begin{figure}[H]
        \centering
        \includegraphics[width=0.75\textwidth]{final_scene.png}
        \caption{Схват с объектом для обнаружения}
        \label{pic:final scene}
    \end{figure}

    \begin{figure}[H]
        \centering
        \includegraphics{final_tree.png}
        \caption{Дерево модели по окончанию работы}
        \label{pic:final tree}
    \end{figure}

    \section*{Вывод}
    В работе были изучены примитивы и простейший сенсор предоставляемые симулятором. Также в процессе
    были освоены базовые принципы языка программирования Lua, на котором пишутся скрипты для описания
    поведения объектов.


    \appendix
    \renewcommand{\thesection}{\Asbuk{section}}
    \section{Управляющий скрипт}\label{code:script}
    \luafile[frame=single]{../src/main.lua}


\end{document}
